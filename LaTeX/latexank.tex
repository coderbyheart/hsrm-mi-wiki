\documentclass[10pt,a4paper]{article}

\usepackage[german]{babel}     %% ich schreibe in Deutsch
\usepackage[utf8]{inputenc}    %% im Zeichensatz utf8
\usepackage[T1]{fontenc}       %% damit die Trennung in deutsch funktioniert
\usepackage{hsrmlogo}          %% das Hochschul-Logo, fast...
\usepackage{verbatim}          %% Anzeigen von Dateien wie eingelesen
% \usepackage{mathptmx}          %% Postscript Times New Roman Zeichensatz
\usepackage{cmlgc}             %% Type 1 fonts

%%% Das Seitenlayout
\setlength{\textwidth}{166mm}
\setlength{\textheight}{210mm}
\setlength{\topmargin}{0mm}
\setlength{\headsep}{10mm}
\setlength{\oddsidemargin}{-6mm}  % Rand aller Seiten einseitig
% Die Kopfzeilen und Fusszeilen 
\usepackage{fancyhdr}
\pagestyle{fancyplain}           % darf erst nach den \setlength Kommandos kommen
\addtolength{\headheight}{5ex} % damit man zwei Zeilen gut unterbringt
\fancyhead[L]{{\sf Hochschule \\ RheinMain}}
\fancyhead[CC]{                  % das HSRM-Logo oben in der Mitte
  \parbox[b]{24pt}{\hsrmlogo[0.5]}
}
\fancyhead[R]{{\sf \LaTeX\--Einführung \\ WS 11/12}} 
\fancyfoot[L]{{\sf Prof.\ Dr.\ Peter Barth \\ Fachbereich Design Informatik Medien}} 
\fancyfoot[C]{\ }
\fancyfoot[R]{{\sf \thepage \\ Medieninformatik}} 

\begin{document}
\begin{center}
\large\bf Einführung in \LaTeX,
27.10.11, 10:00 Uhr, Raum 14
\end{center}

\begin{minipage}{86mm}
\begin{abstract}
Bachelor- oder Seminararbeit schreiben und keine Krise 
kurz vor der Abgabe haben? 
Professionelles Layout und das alles 
ganz automatisch?
\LaTeX\ ist Ihre Wahl.
\end{abstract}

\section{Warum \LaTeX}

\subsection{Dokumente im Buchdruckqualität}

Das von Donald E. Knuth~\cite{bib:Knuth91} entwickelte
Programm \TeX\ ist das leistungsfähigste Programm
zur professionellen Erstellung wissenschaftlicher
und technischer Texte wie Bachelor-Arbeiten und 
Seminararbeiten.

Das auf \TeX\ basierende Paket \LaTeX\ erlaubt es Autoren 
sich auf die Inhalte zu konzentrieren.
Im Gegensatz zu WYSIWYG\footnote{What you see is what you get},
\emph{was eigentlich eher eine Drohung ist},
muss man sich an die Philosopie von \LaTeX\ gewöhnen,
wird aber durch die flexible und stabile Umgebung mit 
hochwertigen Erzeugnissen belohnt. 

\subsection{Features} \label{sec:features}

Natürlich spielt \LaTeX\ seine Stärken gerade im 
Satz mathematischer Formeln wie 
$\sum_{i=1}^{n} i = \frac{n \cdot (n+1)}{2}$ 
aus, die sowohl im Fließtext als auch abgesetzt 
\begin{equation} \label{eq:gauss}
\sum_{i=1}^{n} i = \frac{n \cdot (n+1)}{2}
\end{equation}
wie in Gleichung~(\ref{eq:gauss}) einwandfrei aussehen.

\section{Inhalte}

\begin{itemize}
\item \LaTeX\-Features aus Abschnitt~\ref{sec:features} anhand
eines einfaches Beispiels mit Texmaker unter Windows mit MiKTeX 
oder Linux mit TeXLive.
\item Idee und Grundprinzipien, Dokumente und Seitenstil,
Texthervorhebungen, etc.\enspace.
\end{itemize}

\begin{thebibliography}{1}
\bibitem[Knuth91]{bib:Knuth91}
  Knuth, Donald E..:
  \textit{Computers and Typesetting} Vol. A-E.
  Addison-Wesley Co., Inc., Reading, MA, 1987-1991
\end{thebibliography}
\end{minipage}\mbox{\ \ \ }
\begin{minipage}{77mm}
\footnotesize
\begin{verbatim}
\begin{abstract}
Bachelor- oder Seminararbeit schreiben und 
keine Krise kurz vor der Abgabe haben? 
Professionelles Layout und das alles 
ganz automatisch?
\LaTeX\ ist Ihre Wahl.
\end{abstract}

\section{Warum \LaTeX}

\subsection{Dokumente im Buchdruckqualität}

Das von Donald E. Knuth~\cite{bib:Knuth91} 
entwickelte Programm \TeX\ ist das 
leistungsfähigste Programm zur professionellen 
Erstellung wissenschaftlicher und technischer 
Texte wie Bachelor-Arbeiten und Seminararbeiten.

Das auf \TeX\ basierte Paket \LaTeX\ erlaubt 
es Autoren sich auf die Inhalte zu 
konzentrieren. Im Gegensatz zu 
WYSIWYG\footnote{What you see is what you get},
\emph{was eigentlich eher eine Drohung ist},
muss man sich an die Philosopie von \LaTeX\ 
gewöhnen, wird aber durch die flexible und 
stabile Umgebung mit hochwertigen 
Erzeugnissen belohnt. 

\subsection{Features} \label{sec:features}

Natürlich spielt \LaTeX\ seine Stärken gerade im 
Satz mathematischer Formeln wie 
$\sum_{i=1}^{n} i = \frac{n \cdot (n+1)}{2}$ 
aus, die sowohl im Fließtext als auch abgesetzt 
\begin{equation} \label{eq:gauss}
\sum_{i=1}^{n} i = \frac{n \cdot (n+1)}{2}
\end{equation}
wie in Gleichung~(\ref{eq:gauss}) einwandfrei 
aussehen.


\section{Inhalte}

\begin{itemize}
\item \LaTeX\-Features aus Abschnitt~\ref{sec:features} 
anhand eines einfaches Beispiels mit Texmaker unter 
Windows mit MiKTeX Linux mit TeXLive.
\item Idee und Grundprinzipien, Dokumente und 
Seitenstil, Texthervorhebungen, etc.\enspace.
\end{itemize}



\begin{thebibliography}{1}
\bibitem[Knuth91]{bib:Knuth91}
  Knuth, Donald E..:
  \textit{Computers and Typesetting} Vol. A-E.
  Addison-Wesley Co., Inc., Reading, MA, 1987-1991
\end{thebibliography}

\end{verbatim}
\end{minipage}

\end{document}
